The main goal of this learning project is to explore this relationship.

\[
\begin{tikzcd}[column sep=1.5em, cramped]
	\begin{array}{c} \textcolor{darkblue}{\text{Gromov-Witten Theory}} \\ \textcolor{darkgreen}{\text{of quiver varieties}} \end{array} & {\textcolor{darkred}{\text{Maulik-Okounkov Yangians}}} \\
	\begin{array}{c} \textcolor{darkblue}{\text{Enumerative geometry}} \\ \textcolor{darkgreen}{\text{of quiver representations}} \end{array} & {\textcolor{darkred}{\text{Quantum groups}}} \\
	\begin{array}{c} \textcolor{darkblue}{\text{Donaldson-Thomas Theory}} \\ \textcolor{darkgreen}{\text{of quivers with potential}} \end{array} & {\textcolor{darkred}{\text{Cohomological Hall algebras}}}
	\arrow[from=1-1, to=1-2]
	\arrow[from=2-1, to=1-1]
	\arrow[from=2-1, to=3-1]
	\arrow[from=2-2, to=1-2]
	\arrow[from=2-2, to=3-2]
	\arrow[from=3-1, to=3-2]
\end{tikzcd}
\]

This is more or less an exploration of the intersection of three areas of math: \textcolor{darkblue}{enumerative geometry}, \textcolor{darkgreen}{quivers}, and \textcolor{darkred}{quantum groups}.