\newpage 

\section{Cartan Matrices}

\subsection{Motivation and Definition}

The motivation for caring about Cartan matrices are Kac-Moody Lie algebras, where generalized Cartan matrices serve as the initial building blocks for their construction. Kac-Moody Lie algebras, particularly of the affine type, are a key ingredient in several fields of physics. 
\begin{itemize}
    \item \textbf{Conformal field theory}: Affine Kac-Moody algebras provide the symmetry structure necessary to construct conformal blocks and to implement the modular invariance of partition functions in conformal field theory.
    \item \textbf{String theory}: Affine Kac-Moody Lie algebras provide the algebraic structure for symmetry transformations and vertex operator algebras, which are essential for the formulation of consistent string interactions and compactifications.
\end{itemize}
Although these are the two main applications, Kac-Moody Lie algebras also appear in integrable systems, statistical mechanics, gauge theories, and more. See \cite{KMPHYS} for more.

Now that we know about why we should care about Cartan matrices, let's define a Cartan matrix.

\begin{definition}
    An $n\times n$ matrix $C=(c_{i,j})$ with integer entries $c_{i,j}$ is called a \textbf{Cartan matrix} if:
    \begin{itemize}
        \item $c_{i,i}=2$ for all $i$
        \item $c_{i,j}\le 0$ for all $i\neq j$
        \item $c_{i,j}=0$ if and only if $c_{j,i}$, for all $i,j$.
    \end{itemize}
\end{definition}

Most of the time, we will assume that $C$ is symmetrizable: that is, there exists a diagonal matrix $D=\text{diag}(d_1,...,d_n)$ with $d_i$ positive integers such that $DC$ is symmetric.

Now, we present two different ways in which Cartan matrices naturally arise.

\subsection{Finite Graphs with Automorphisms}

The point of this realization is to motivate studying the representation theory of quivers with automorphisms.

We can encode all of the information of a Cartan matrix inside a valued graph, which can be then encoded in a graph with automorphism, such that there is no loss of information in this process.

Let's review some standard graph theory terminology: Consider a graph $\Sigma=(V,E,f)$ where $v\in V$ are the vertices, $e\in E$ are the edges, and a function $f(e)=\{u,v\}$. We allow loops as edges. An isomorphism $\Sigma\to \Sigma'$ consists of a pair of bijections $\sigma: V\to V', E\to E'$ such that if $f(e)=\{u,v\}$, then $f'(\sigma(e))=\{\sigma(u),\sigma(v)\}$. An automorphism is an isomorphism $\Sigma\to \Sigma$.

Now, we define a valued graph $\Gamma$ as a graph $\Sigma$ with positive integers $d_v$ or $m_e$ assigned to each vertex or edge such that $m_e$ is a common multiple of $d_u$ and $d_v$. Two valued graphs are isomorphic if there is an underlying graph isomorphism such that the values attached to $v$ and $\sigma(v)$ and to $e$ and $\sigma(e)$ are the same for all $v,e$. The values $d_u$ and $m_u$ attached to $u$ or $v$ of $\Gamma$ is the cardinality of the corresponding orbit. 

\begin{definition}
    A graph with automorphism $(\Sigma,\sigma)$ is a \textbf{graph realization} of $C$ if $\Gamma(\Sigma, \sigma ) \cong \Gamma(C)$.
\end{definition}

\begin{theorem}
    Every $C$ has a graph realization.
\end{theorem}

\begin{proof}
We perform the following procedure.

\[\begin{tikzcd}
	{(\Sigma, \sigma)} && {\Gamma(\Sigma, \sigma)} \\
	\\
	C && {\Gamma(C)} && {(\Sigma(C), \sigma(C))}
	\arrow["1", from=1-1, to=1-3]
	\arrow["4", no head, from=1-3, to=3-3]
	\arrow["2", from=3-1, to=3-3]
	\arrow["3", from=3-3, to=3-5]
\end{tikzcd}\]

\begin{enumerate}
    \item Given $(\Sigma, \sigma)$, we construct $\Gamma(\Sigma, \sigma)$ through the folding procedure: Assign a graph with automorphism $(\Sigma, \sigma)$ a valued graph $\Gamma(\Sigma, \sigma)$: the vertices and edges are the $\sigma$-orbits of the vertices and edges in $\Sigma$.
    \item Given $C$, we construct $\Gamma(C)$ as follows.
    
    The underlying graph: the vertices are $1,...,n$. If $i\neq j$, then there are $n_{i,j}=-\frac{d_i c_{i,j}}{d_{i,j}}$ parallel edges connecting $i$ and $j$, where $d_{i,j}=\text{lcm}(d_i, d_j)$. 
    
    The weights: Each vertex $i$ is assigned the weight $d_i$, and each edge is assigned the weight $m_e:=d_{i,j}$.
    \item Given $\Sigma(C)$, we construct $(\Sigma(C), \sigma(C))$.
    
    The underlying graph: There are $d_1+...+d_n$ vertices $v_{i,k}$ for $1\le i \le n,k \in \mathbb{Z}/(d_i)$, and for fixed $i,j$ there are $n_{i,j}$ parallel edges $e_{k,l}^{(a)}$ for $1\le a \le n_{i,j}$ connecting $v_{i,k}$ with $v_{j,l}$, where $(k,l)$ runs over the cyclic subgroup $\langle (1,1)\rangle$ of the abelian group $\mathbb{Z}/(d_i) \times \mathbb{Z}/(d_j)$ generated by $(1,1)$.

    The automorphism: for every $i, \sigma$ cyclicly permutes the vertices $v_{i,1},...,v_{i,d_i}$. This is clearly an automorphism.
    \item By construction, this is an isomorphism.
\end{enumerate}

\end{proof}

To get a better sense of this construction, let's compute one example explicitly:

\begin{example}
    Consider 
    \[C=\begin{pmatrix}
        2 & -4 \\ -6 & 2
    \end{pmatrix}, D= \begin{pmatrix}
        3 & 0 \\ 0 & 2
    \end{pmatrix}. \] 

    \begin{itemize} 
        \item Construction of $\Gamma(C)$:
        $C$ is a $2\times 2$ matrix, so $n=2$. Thus, we have two vertices, $1$ and $2$. The weight on vertex $1$ is $d_1=3$, and the weight on vertex $2$ is $d_2=2$.
        
        The number of edges between the two vertices is:
        \[n_{1,2}=-\frac{d_1 c_{1,2}}{\text{lcm}(d_1, d_2)} = -\frac{3\cdot -4}{\text{lcm}(2,3)} =2. \]
        We can check this:
        \[n_{2,1}=-\frac{d_2 c_{2,1}}{\text{lcm}(d_2, d_1)} = -\frac{2\cdot -6}{\text{lcm}(3,2)} =2. \]
        The weights on the edges are $m_e:= d_{i,j}= \text{lcm}(2,3)=6$.

        This gives 
        \[\begin{tikzcd}
            3 &&& 2
            \arrow["6"', curve={height=12pt}, no head, from=1-1, to=1-4]
            \arrow["6", curve={height=-12pt}, no head, from=1-1, to=1-4]
        \end{tikzcd}\] 

        \item Construction of $(\Sigma(C), \sigma(C))$: There are $d_1+d_2=5$ vertices and each vertex coming from $d_1$ has $n_{i,j}=2$ edges with each vertex coming from $d_2$:
        
        \[\begin{tikzcd}
            \bullet \\
            &&&&& \bullet \\
            \bullet \\
            &&&&& \bullet \\
            \bullet
            \arrow[dotted, no head, from=1-1, to=2-6]
            \arrow[curve={height=6pt}, no head, from=1-1, to=2-6]
            \arrow[dotted, no head, from=1-1, to=4-6]
            \arrow[curve={height=6pt}, no head, from=1-1, to=4-6]
            \arrow[dotted, no head, from=3-1, to=2-6]
            \arrow[curve={height=6pt}, no head, from=3-1, to=2-6]
            \arrow[dotted, no head, from=3-1, to=4-6]
            \arrow[curve={height=6pt}, no head, from=3-1, to=4-6]
            \arrow[dotted, no head, from=5-1, to=2-6]
            \arrow[curve={height=6pt}, no head, from=5-1, to=2-6]
            \arrow[dotted, no head, from=5-1, to=4-6]
            \arrow[curve={height=6pt}, no head, from=5-1, to=4-6]
        \end{tikzcd}\]
        
    \end{itemize}

\end{example}

\begin{definition}
    A Cartan matrix $C$ is called \textbf{indecomposable} if the underlying graph of $\Gamma(C)$ is a connected graph.
\end{definition}

Through lots of messy linear algebra, it can be shown that after renumbering the rows/columns, every Cartan matrix can be uniquely decomposed into block diagonal form such that each block is a Cartan matrix.

\subsection{Root Datum Realization}

The point of this realization is the Lie theory/quantum groups side of the story.

\begin{definition}
    Let $C=(c_{i,j})$ be an $n\times n$ Cartan matrix. A 4-tuple $\mathfrak{R}= (\Pi, X, \Pi^\vee, X^\vee)$ is called a \textbf{root datum realization} of $C=(c_{i,j})$ if 
    \begin{itemize}
        \item $X^\vee$ is a free $\mathbb{Z}$-module of finite rank $n+m$, for some $m\in \mathbb{N}$, having an ordered basis $\{\alpha_1^\vee, ..., \alpha_n^\vee, b_1,...,b_m\}$. 
        \item $X:= \text{Hom}(X^\vee,\mathbb{Z})$ is the $\mathbb{Z}$-linear dual of $X^\vee$, with the duality pairing 
        \[X\times X^\vee \to \mathbb{Z}, \qquad (\alpha, h) \mapsto \langle \alpha, h \rangle = \alpha(h)\]
        \item $\Pi^\vee := \{\alpha_1^\vee, ..., \alpha_n^\vee \}$ whose elements are called \textbf{simple coroots}
        \item $\Pi = \{\alpha_1,...,\alpha_n\}$ is a fixed subset of $X$ satisfying 
        \begin{itemize}
            \item $\langle \alpha_i, \alpha_j^\vee \rangle = c_{j,i}$ for all $i,j$
            \item $a_{j,i} = \langle \alpha_i, b_j\rangle$ for all $i,j$ such that the combined $(n+m)\times n$ matrix $\begin{pmatrix}
            C \\ A
            \end{pmatrix}$, where $A=(a_{i,j})$, has rank $n$.
        \end{itemize}
        whose elements are linearly independent and are called \textbf{simple roots}.
    \end{itemize}
\end{definition}

By linear algebra, it is not hard to see that a realization exists if and only if $m\ge n-\text{rank}(C)$.

\begin{definition}
    The \textbf{root lattice} of $\mathfrak{R}$ is 
    \[R(\Pi) := \mathbb{Z}\alpha_1 \oplus ... \oplus \mathbb{Z}\alpha_n \subseteq X \]     
    where $X$ is a free $\mathbb{Z}$-module of rank $n+m$ called the \textbf{weight lattice} and the \textbf{fundamental dominant weights} are the vectors $\overline{\omega_1},...,\overline{\omega_n}\in X$ defined by  
    \[\langle \overline{\omega_i}, \alpha_j^\vee \rangle = \delta_{i,j}, \qquad \langle \overline{\omega_i}, b_t\rangle =0 \qquad \text{ for }1\le i,j\le n \text{ and } 1\le t\le m.\]
\end{definition}

\begin{definition}
    A weight $\lambda \in X$ is \textbf{dominant} if $\langle \lambda, \alpha_j^\vee \rangle \ge 0$ for $1\le j \le n$.
\end{definition}

\begin{definition}
    The \textbf{Weyl group} $W(C)$ is the subgroup of $\text{GL}(\mathfrak{h}_{\mathbb{R}}^*)$ generated by $s_i$, where 
    \begin{itemize}
        \item $\mathfrak{h}_{\mathbb{R}}^*$ is the $\mathbb{R}$-linear dual of $\mathfrak{h}_{\mathbb{R}} = X^\vee \otimes_{\mathbb{Z}} \mathbb{R}$ 
        \item $s_i\in \text{GL}(\mathfrak{h}_\mathbb{R}^*)$ are the linear transformations defined by
        \[s_i(\zeta) = \zeta - \langle \zeta,\alpha_i^\vee \rangle \alpha_i \]
        for $\zeta\in \mathfrak{h}_{\mathbb{R}}^*$.
    \end{itemize}
\end{definition}

It's not hard to see that $W(C)$ stabilizes $X$, and the action $W(C)$ on $X$ induces a contragradient action of $W(C)$ on $X^\vee$ satisfying for $\zeta \in X, h\in X^\vee$,
\[\langle \zeta, w(h)\rangle = \langle w^{-1}(\zeta), h\rangle.\]

\section{}